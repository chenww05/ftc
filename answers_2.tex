\documentclass{letter}
\usepackage{graphicx}
\usepackage{color}
\date{Mar 17, 2015}

\newenvironment{review}%          environment name
{\textbf{Reviewer comment:}\begin{quote}}% begin code
{\end{quote}}%  

\newcommand{\todo}[1]{%                                                                                                                            
      \color{red}\textbf{[TODO]} #1\color{black}}

\newcommand{\answer}[1]{%                                                                                                                            
      \textbf{Answer:} #1}

\usepackage{hyperref}

\newcommand{\revised}[1]{\emph{#1}\color{black}}
\newcommand{\rev}[1]{\color{blue} #1\color{black}}

\begin{document}

\begin{letter}{}

\opening{Dear Editor and Reviewers,}

We would like to thank you for your thorough review and the useful
comments and suggestions formulated about our paper ``Dynamic and Fault-Tolerant Clustering for Scientific Workflows'' submitted to
IEEE Transactions on Cloud Computing.

The remainder of this letter contains our answers to your reviews. 

We thank you again for your valuable review, and we remain available for any further information you may need.

\vspace{0.5cm}

Sincerely yours,

\vspace{1cm}

Weiwei Chen, Rafael Ferreira da Silva, Ewa Deelman, Thomas Fahringer

\newpage

%
% Reviewer 1
%
\textbf{Reviewer 3}

\begin{review}
I had thought that to consider the network bandwidth competition can better improve the fidelity of the simulation. Given the added assumption in the revision. I think that could be put to the future work. Other than this, I think my previous comments are well addressed in the revision.
\end{review}

\answer{We updated our Future Work to include other network bandwidth models:}

\revised{In this paper we assumed that the network 
bandwidth is the maximum allowed data transfer speed between a pair of virtual machines per file. 
Future work will elaborate different network models to explore their impact on our fault-tolerant 
clustering techniques.}


\end{letter}
\end{document}